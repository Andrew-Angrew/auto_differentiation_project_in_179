\documentclass[12pt,a4paper]{article}

\usepackage[T2A]{fontenc} % Поддержка русских букв
\usepackage[utf8]{inputenc} % Кодировка utf8
\usepackage[english, russian]{babel} % Языки: русский, английский
% \usepackage{pscyr} % Нормальные шрифты

\usepackage{amsmath, amssymb}
\usepackage{amscd}
\usepackage{amsthm}
\usepackage{graphicx}
\usepackage[left=2cm,right=2cm,top=2cm,bottom=1.5cm]{geometry}
\usepackage{fancyhdr}
%\parindent = 0em %расстояние от края в красных строках

\usepackage{graphicx}
\graphicspath{ {./pic/} }

% чтобы itemize работал без лишних пробелов
\usepackage{enumitem}
\setlist{nolistsep,
	% itemsep=0.3cm,
	% parsep=4pt
}

\def\Z{\mathbb{Z}}
\def\N{\mathbb{N}}
\def\R{\mathbb{R}}

\newcounter{znum}
\newcommand{\z}[1]{\addtocounter{znum}{1} \textbf{Задача \arabic{znum}#1. }}

\newcounter{defnum}
\newcommand{\df}[1]{\addtocounter{defnum}{1} \textbf{Определение \arabic{defnum}.} {\it #1}}

\newcommand{\zp}[1]{\par~~~~\textbf{#1)}}

\usepackage{xcolor}
\usepackage{hyperref}
\definecolor{linkcolor}{HTML}{799B03} % цвет ссылок
%\definecolor{urlcolor}{HTML}{799B03} % цвет гиперссылок
\hypersetup{
	pdfstartview=FitH,
	linkcolor=linkcolor,
	urlcolor=blue,
	colorlinks=true
}

\begin{document}
\pagestyle{empty}

\begin{center} \Large \textbf{Производные на скорую руку.}
\end{center}

{\it {\bf "о" малое.} \href{https://ru.wikipedia.org/wiki/\%C2\%ABO\%C2\%BB\_\%D0\%B1\%D0\%BE\%D0\%BB\%D1\%8C\%D1\%88\%D0\%BE\%D0\%B5\_\%D0\%B8\_\%C2\%ABo\%C2\%BB\_\%D0\%BC\%D0\%B0\%D0\%BB\%D0\%BE\%D0\%B5}{читайте тут.}}



В этом листке для двух функций $f$ и $g$ будут встречаться записи вида $f(\varepsilon) = o(g(\varepsilon))$. Они означают, что для любой бесконечно малой последовательности $(\varepsilon_n)$ ненулевых чисел, последовательность $\frac{f(\varepsilon_n)}{|g(\varepsilon_n)|}$ -- бесконечно малая. При этом предполагается, что $g(\varepsilon) \ne 0$ при $\varepsilon \ne 0$.
Подробнее про такие обозначения можно почитать в третьей главе Кормена.

{\it {\bf Определение 2.} Число $a$ называют производной функции $f$ в точке $x$, если
$$f(x + \varepsilon) = f(x) + a \varepsilon + o(\varepsilon)$$
Иными словами $f(x + \varepsilon) = f(x) + a \varepsilon + g(\varepsilon)$, где $g(\varepsilon) = o(\varepsilon)$.}

Производная функции $f$ в точке $x$ обозначается $f'(x)$

\z{} Осмыслите и проверьте следующие тождества:\par
\textbf{а)} $\varepsilon^2 = o(\varepsilon)$,\par
\textbf{б)} $\varepsilon^3 = o(\varepsilon^2)$,\par
\textbf{в)} $o(\varepsilon^3) + o(\varepsilon^2) = o(\varepsilon^2)$,\par
\textbf{г)} $\varepsilon = o(1)$,\par
\textbf{д)} $\varepsilon \cdot o(\varepsilon) = o(\varepsilon^2)$.

\z{} Докажите эквивалентность определений производной.

\z{} Пусть функции $f$ и $g$ дифференцируемы в точке $x$. Найдите производные в точке $x$ у функций $f + g$ и $f \cdot g$ (и докажите, что они существуют).

\z{} Пусть функция $f$ дифференцируема в точке $x$, а $g$ -- в точке $f(x)$. \textbf{а)}~Найдите производную $g(f(x))$ в точке $x$. \textbf{а)} Пусть также $f$ и $g$ взаимно обратны, т. е. $g(f(x)) = x$. Докажите, что $g'(f(x)) = \frac{1}{f'(x)}$.

\z{} Докажите, что $\frac{1}{1 + \varepsilon} = 1 - \varepsilon + o(\varepsilon)$.

\z{} Найдите производную функции $\frac1x$ при $x \ne 0$.

\z{} Пусть функции $f$ и $g$ дифференцируемы в точке $x$ и $g(x) \ne 0$. Найдите производную $\frac{f(x)}{g(x)}$ в точке $x$.

\z{} Найдите производные следующих функций:\par
\textbf{а)}~$sin$\par
\textbf{б)}~$cos$\par
\textbf{в)}~$x^n$ ($n \in \N$)\par
\textbf{г)}~$x^{-n}$ ($n \in \N$)

Экспонента -- это функция $\exp(x)$, такая что $\exp'(x) = \exp(x)$ и $\exp(0) = 1$. $\exp(x)$ также записывают как $e^{x}$. Логарифм ($\log(x)$) -- это обратная функция к экспоненте, определенная только на положительных числах. Т. е. $\forall x : \log(\exp(x)) = x$ и $\forall x > 0 : \exp(\log(x)) = x$.
То, что экспонента и логарифм существуют и дифференцируемы на всей своей области определения давайте пока считать очевидным.

\z{} Найдите производную $log(x)$ (при $x > 0$).

\z{* (ненужная)} Помахайте руками и обоснуйте тождество:
$$\exp(a + b) = \exp(a) \cdot \exp(b)$$

\end{document}
