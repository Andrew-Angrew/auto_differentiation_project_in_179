\documentclass[12pt,a4paper]{article}

\usepackage[T2A]{fontenc} % Поддержка русских букв
\usepackage[utf8]{inputenc} % Кодировка utf8
\usepackage[english, russian]{babel} % Языки: русский, английский
% \usepackage{pscyr} % Нормальные шрифты

\usepackage{amsmath, amssymb}
\usepackage{amscd}
\usepackage{amsthm}
\usepackage{graphicx}
\usepackage[left=1.5cm,right=1.5cm,top=2cm,bottom=1.5cm]{geometry}
\usepackage{fancyhdr}
%\parindent = 0em %расстояние от края в красных строках

\usepackage{graphicx}
\graphicspath{ {./pic/} }

%для ссылок
\usepackage{xcolor}
\usepackage{hyperref}
\definecolor{linkcolor}{HTML}{799B03} % цвет ссылок
%\definecolor{urlcolor}{HTML}{799B03} % цвет гиперссылок
\hypersetup{pdfstartview=FitH,  linkcolor=linkcolor, urlcolor=blue
	, colorlinks=true}


% чтобы itemize работал без лишних пробелов
\usepackage{enumitem}
\setlist{nolistsep,
	% itemsep=0.3cm,
	% parsep=4pt
}

\def\Z{\mathbb{Z}}
\def\N{\mathbb{N}}
\def\R{\mathbb{R}}
\def\E{\mathbb{E}}
\def\D{\mathbb{D}}

\newcounter{znum}
\newcommand{\zz}[1]{\addtocounter{znum}{1} \textbf{Задача \arabic{znum}#1. }}
\newcommand{\z}[1]{\addtocounter{znum}{1} Задача \arabic{znum}#1. }

\newcounter{defnum}
\newcommand{\df}[1]{\addtocounter{defnum}{1} \textbf{Определение \arabic{defnum}.} {\it #1}}

\newcommand{\zp}[1]{\par~~~~\textbf{#1)}}

\begin{document}
\pagestyle{empty}

\begin{center} \Large \textbf{Градиент.}
\end{center}

Любую функцию от $n$ переменных можно считать функцией из $\R^n$ в $\R$. 
Пусть у нас есть величина $q(\mathbf{u})$, зависящая от вектора $\mathbf{u} \in \R^n$. Будем говорить, что $q(\mathbf{u}) = o(\mathbf{u})$ (читается "$q(\mathbf{u})$ есть о маленькое от $\mathbf{u}$"), если для любой последовательности векторов $\mathbf{u}_1, \mathbf{u}_2, \ldots \in \R$ такой, что $\|\mathbf{u}_i\| \to 0$, последовательность $q(\mathbf{u}_i)$ стремится к нулю.


{\it Функция $f$ от $n$ переменных \textbf{\href{https://ru.wikipedia.org/wiki/\%D0\%94\%D0\%B8\%D1\%84\%D1\%84\%D0\%B5\%D1\%80\%D0\%B5\%D0\%BD\%D1\%86\%D0\%B8\%D1\%80\%D1\%83\%D0\%B5\%D0\%BC\%D0\%B0\%D1\%8F_\%D1\%84\%D1\%83\%D0\%BD\%D0\%BA\%D1\%86\%D0\%B8\%D1\%8F\#\%D0\%A4\%D1\%83\%D0\%BD\%D0\%BA\%D1\%86\%D0\%B8\%D0\%B8_\%D0\%BD\%D0\%B5\%D1\%81\%D0\%BA\%D0\%BE\%D0\%BB\%D1\%8C\%D0\%BA\%D0\%B8\%D1\%85_\%D0\%BF\%D0\%B5\%D1\%80\%D0\%B5\%D0\%BC\%D0\%B5\%D0\%BD\%D0\%BD\%D1\%8B\%D1\%85}{дифференцируема}} в точке $\mathbf{x}$, если существует такой вектор $\nabla f(\mathbf{x}) \in \R^n$, что для любого $v \in \R^n$
$$ f(\mathbf{x} + \mathbf{v}) = f(\mathbf{x}) + \langle \mathbf{v}, \nabla f(\mathbf{x}) \rangle + o(\mathbf{v})$$
Вектор $\nabla f(\mathbf{x})$ называется \textbf{\href{https://ru.wikipedia.org/wiki/\%D0\%93\%D1\%80\%D0\%B0\%D0\%B4\%D0\%B8\%D0\%B5\%D0\%BD\%D1\%82}{градиентом}} функции $f$ в точке $\mathbf{x}$.
}

\zz{} Убедитесь, что для $n=1$ новое определение дифференцируемости совпадает с уже знакомым определением дифференцируемости функции из $\R$ в $\R$.

\zz{} Пусть $f$ и $g$ -- дифференцируемые функции из $\R$ в $\R$ и из $\R^n$ в $\R$ соответственно. Докажите, что $f(g(\mathbf{x}))$ -- дифференцируемая функция из $\R^n$  в $\R$.

\zz{} Пусть $f$ и $g$ -- дифференцируемые функции из $\R^n$ в $\R$. Докажите, что следующие функции дифференцируемы (и выразите их градиенты, через градиенты $f$~и~$g$):\par
\textbf{a)} $f + g$ \par
\textbf{б)} $f \cdot g$ \par
\textbf{в)} $f / g$ (в точках $\mathbf{x}$, где $g(\mathbf{x}) \ne 0$)

\zz{} Докажите, что $f(g_1(\mathbf{x}), \ldots, g_m(\mathbf{x}))$ -- всюду дифференцируемая функция из $\R^n$ в $\R$, если $g_i: \R^n \to \R$ и $f: \R^m \to \R$ всюду дифференцируемы.

\z{} Докажите, что следующие функции $\R^n \to \R$ дифференцируемы по $\mathbf{x}$: \par
a) $x_1 \cdot \ldots \cdot x_n$ \par
б) $\sin(x_1 + \ldots + x_n)$ \par
\textbf{в)} $\log(\sigma(\langle \mathbf{x}, \mathbf{w} \rangle))$, где $\sigma(x) = \frac{1}{1 + e^{-x}} = \frac{e^{x}}{1 + e^{x}}.$\par
\textbf{г)} И эта функция $\R^n \to \R^n$: $\mathrm{softmax}(\mathbf{x}) = \frac{(e^{x_1}, \ldots, e^{x_n})}{e^{x_1} + \ldots + e^{x_n}}$\par

{\it \textbf{Частная производная} функции $f : \R^n \to \R$ по $i$-той переменной в точке $\mathbf{x}$ это:
$$ \frac{\partial f(\mathbf{x})}{\partial x_i} = \lim_{\varepsilon \to 0} \frac{f(x_1, \ldots, x_i + \varepsilon,\ldots, x_n) - f(x_1, \ldots, x_i,\ldots, x_n)}{\varepsilon} $$
Иными словами мы фиксируем все переменные кроме $i$-той, рассматриваем $f$ как функцию от одной переменной и берем ее производную в точке $x_i$.
}

\zz{} Пусть $f : \R^n \to \R$ дифференцируема в точке $\mathbf{x}$. Докажите что:
$$ \nabla f(\mathbf{x}) = \left(\frac{\partial f(\mathbf{x})}{\partial x_1}, \ldots, \frac{\partial f(\mathbf{x})}{\partial x_n}\right) .$$

\zz{} Пусть мы находимся в точке $\mathbf{a}$, $\nabla f(\mathbf{a}) = \mathbf{g} \ne \overrightarrow{\mathbf{0}}$
 и $\| \mathbf{v} \|$. Предроло $ = \delta \to 0$.

\textbf{а)} Убедитесь что: $ f(\mathbf{a} + \mathbf{v}) = f(\mathbf{a}) + \langle \mathbf{v}, \mathbf{g} \rangle + o(\delta)$\par
\textbf{б)} Для любого вектора $\mathbf{v}$ обозначим через $\mathbf{v}_\delta$ вектор $\delta \cdot \frac{\mathbf{g}}{\|\mathbf{g}\|}$ (шаг в направлении $\mathbf{v}$ длины $\delta$). Попробуем сдвинуться на $\delta$ так, чтобы функция $f$ возросла как можно больше. Докажите, что для любого направления $\mathbf{v}$, непропорционального градиенту $\mathbf{g}$,  для достаточно маленького $\delta$ $f(\mathbf{a} + \mathbf{v}_\delta) < f(\mathbf{a} + \mathbf{g}_\delta)$. Таким образом градиент указывает направление вдоль которого функция возрастает максимально быстро.

\zz{} Докажите, что в локальном оптимуме градиент равен $\overrightarrow{\mathbf{0}}$.

\df{\textbf{Градиентный спуск} -- это процесс ''жадной'' минимизации функции, действующий по правилу вида $\mathbf{x_{t + 1}} = \mathbf{x_{t}} - \lambda \cdot \nabla f(\mathbf{x_{t}})$. Здесь $\lambda$ -- это параметр называемый \textbf{learning rate}. Он как правило меняется от точки к точке (в зависимости от конкретной реализации).}

\z{*} Докажите, что $f$ дифференцируема в точке $\mathbf{x}$, если ее частные производные определены в некоторой окрестности $\mathbf{x}$ и непрерывны в $\mathbf{x}$.


\end{document}
