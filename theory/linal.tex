\documentclass[12pt,a4paper]{article}

\usepackage[T2A]{fontenc} % Поддержка русских букв
\usepackage[utf8]{inputenc} % Кодировка utf8
\usepackage[english, russian]{babel} % Языки: русский, английский
% \usepackage{pscyr} % Нормальные шрифты

\usepackage{amsmath, amssymb}
\usepackage{amscd}
\usepackage{amsthm}
\usepackage{graphicx}
\usepackage[left=1.5cm,right=1.5cm,top=2cm,bottom=1.5cm]{geometry}
\usepackage{fancyhdr}
%\parindent = 0em %расстояние от края в красных строках

\usepackage{graphicx}
\graphicspath{ {./pic/} }

%для ссылок
\usepackage{xcolor}
\usepackage{hyperref}
\definecolor{linkcolor}{HTML}{799B03} % цвет ссылок
%\definecolor{urlcolor}{HTML}{799B03} % цвет гиперссылок
\hypersetup{pdfstartview=FitH,  linkcolor=linkcolor, urlcolor=blue
	, colorlinks=true}


% чтобы itemize работал без лишних пробелов
\usepackage{enumitem}
\setlist{nolistsep,
	% itemsep=0.3cm,
	% parsep=4pt
}

\def\Z{\mathbb{Z}}
\def\N{\mathbb{N}}
\def\R{\mathbb{R}}
\def\E{\mathbb{E}}
\def\D{\mathbb{D}}

\newcounter{znum}
\newcommand{\zz}[1]{\addtocounter{znum}{1} \textbf{Задача \arabic{znum}#1. }}
\newcommand{\z}[1]{\addtocounter{znum}{1} Задача \arabic{znum}#1. }

\newcounter{defnum}
\newcommand{\df}[1]{\addtocounter{defnum}{1} \textbf{Определение \arabic{defnum}.} {\it #1}}

\newcommand{\zp}[1]{\par~~~~\textbf{#1)}}

\begin{document}
\pagestyle{empty}

\begin{center} \Large \textbf{Линал++.}
\end{center}

Вектора из $\R^n$ будем обозначать жирными буквами, а значения отдельных чисел в векторе -- той же буквой, только нежирной и с индексом. Например $x_i$ -- это $i$-тое число в векторе $\mathbf{x}$. \textbf{Жирные} задачи важны для понимания дальнейшего материала, а не жирные просто к слову пришлись.

{\it \textbf{Определение 1. Cкалярное произведение} на $\R^n$ это любая функция $\langle \cdot , \cdot \rangle$ от двух векторов такая что:\par
\begin{itemize}
	\item $\langle \mathbf{x}, \mathbf{y} \rangle = \langle \mathbf{y}, \mathbf{x} \rangle$ (симметричность)
	\item $\langle \mathbf{x}_1 + \mathbf{x}_2, \mathbf{y} \rangle = \langle \mathbf{x}_1, \mathbf{y} \rangle + \langle \mathbf{x}_2, \mathbf{y} \rangle$  и $\langle c \cdot \mathbf{x}, \mathbf{y} \rangle = c \cdot \langle \mathbf{x}, \mathbf{y} \rangle$ (билинейность)
	\item $\langle \mathbf{x}, \mathbf{x} \rangle \geqslant 0$ при этом  $\langle \mathbf{x}, \mathbf{x} \rangle = 0 \Leftrightarrow \mathbf{x} = \overrightarrow{\mathbf{0}}$ (положительная определенность)
\end{itemize}
}

\zz{} Проверьте, что операция $ \langle \mathbf{x}, \mathbf{y} \rangle = \sum_{i = 1}^n x_i \cdot y_i$ является скалярным произведением. Ее называют {\it стандартным скалярным произведением}.

\zz{} Проверьте, что скалярное произведение векторов на плоскости из геометрии (произведение длин на косинус) является скалярным произведением в смысле определения 1.

\z{} С легкостью (и \href{https://www.youtube.com/watch?v=NpYzlF1c\_1I}{алгебраично!}) выведите теорему косинусов из предыдущей задачи.

{\it \textbf{Определение 2. Норма} (она же длина) вектора $\mathbf{x} \in \R^n$ это $ \|\mathbf{x}\| = \sqrt{\langle \mathbf{x}, \mathbf{x} \rangle}$, (заметьте, что определение нормы зависит от выбора скалярного произведения)}

\z{ (многомерная теорема Пифагора)} а) Пусть вектора $\mathbf{x}$ и $\mathbf{y}$ ортогональны, то есть $\langle \mathbf{x}, \mathbf{y} \rangle = 0$. Убедитесь что $\|\mathbf{x} + \mathbf{y}\| = \sqrt{\|\mathbf{x}\|^2 + \|\mathbf{y}\|^2}$.

б) Пусть есть много попарно ортогональных векторов $\mathbf{x_1}, \mathbf{x_2}, \ldots \mathbf{x_k}$. Докажите что: $$\|\mathbf{x_1} + \mathbf{x_1} + \ldots + \mathbf{x_k}\| = \sqrt{\|\mathbf{x_1}\|^2 + \|\mathbf{x_2}\|^2 + \ldots + \|\mathbf{x_k}\|^2}. $$

\zz{} Пусть $\langle \cdot , \cdot \rangle$ -- (произвольное) скалярное произведение на плоскости. \par
\textbf{а)} Докажите, что есть базис $(v_1, v_2)$, такой что $\langle v_1, v_2 \rangle = 0$ и $\|v_1\| = \|v_2\| = 1$ (такой базис называется {\it \textbf{ортонормальным}}).\par
\textbf{б)} Докажите, что в системе координат, заданной ортонормальным базисом  $\langle \cdot , \cdot \rangle$ является \underline{стандартным} скалярным произведением.\par
в) Обобщите результат предыдущих пунктов на $\R^n$.\par
г) Выведите из \textbf{б)} эквивалентность двух определений скалярного произведения на плоскости (через косинус и через координаты).

{\it \textbf{Определение 3.} Функция $l: \R^ n \to \R$ называется \textbf{линейной}, если:
	\begin{itemize}
		\item $\forall \mathbf{x} \in \R^n, c \in \R : l(c \cdot \mathbf{x}) = c \cdot l(\mathbf{x})$
		\item $\forall \mathbf{x}, \mathbf{y} \in \R^n : l(\mathbf{x} + \mathbf{y}) = l(\mathbf{x}) + l(\mathbf{y})$
	\end{itemize}
}

\zz{} Докажите, что линейность функции $l$ равносильна следующему:
$$ \exists \mathbf{v} \in \R^n : l(\mathbf{x}) = \langle \mathbf{x}, \mathbf{v}\rangle.$$


\zz{ (неравенство Коши-Буняковского-Шварца)} Докажите\footnote{
	Есть еще \href{https://ru.wikipedia.org/wiki/\%D0\%9D\%D0\%B5\%D1\%80\%D0\%B0\%D0\%B2\%D0\%B5\%D0\%BD\%D1\%81\%D1\%82\%D0\%B2\%D0\%BE\_\%D0\%9A\%D0\%BE\%D1\%88\%D0\%B8\_\%E2\%80\%94\_\%D0\%91\%D1\%83\%D0\%BD\%D1\%8F\%D0\%BA\%D0\%BE\%D0\%B2\%D1\%81\%D0\%BA\%D0\%BE\%D0\%B3\%D0\%BE\#\%D0\%94\%D0\%BE\%D0\%BA\%D0\%B0\%D0\%B7\%D0\%B0\%D1\%82\%D0\%B5\%D0\%BB\%D1\%8C\%D1\%81\%D1\%82\%D0\%B2\%D0\%BE}{стандартное доказательство} и 
	\href{https://zftsh.online/public/folder\_attachment/a0/a9/a8f7\_061b.pdf?c=006f}{мерзкое~через~буковки}.} что :
$$ \langle \mathbf{x}, \mathbf{y} \rangle  \leqslant \|\mathbf{x}\| \cdot \|\mathbf{y}\|$$\par
\textbf{а)} для $\R^2$ (через \textbf{5а, б})\par
\textbf{б)} для $\R^n$ (через пункт \textbf{а})\par
в) для любых (возможно бесконечномерных) векторных пространств над $\R$.\par
\zz{} Равенство в КБШ достигается тогда и только тогда, когда вектора $\mathbf{x}$ и $\mathbf{y}$ пропорциональны.

\zz{} Проверьте свойства нормы:\par
\textbf{а)} $\|\mathbf{x}\| \geqslant 0$, при этом $\|\mathbf{x}\| = 0 \Leftrightarrow \mathbf{x} = \overrightarrow{\mathbf{0}}$.\nopagebreak\par
\textbf{б)} $\|c \cdot \mathbf{x}\| = |c| \cdot \|\mathbf{x}\|$.\nopagebreak \par
\textbf{в)} (многомерное неравенство треугольника) $\|\mathbf{x} + \mathbf{y}\| \leqslant \|\mathbf{x}\| + \|\mathbf{y}\|$ (используйте КБШ).

\z{} Пусть $a + b + c = 1$. Докажите, что $a^2 + b^2 + c^2 \geqslant \frac13$

\z{} Пусть $a_1,\ldots, a_n > 0$. Докажите, что $(a_1 + \ldots + a_n) \left(\frac1{a_1} + \ldots + \frac1{a_n}\right) \geqslant n^2$

{\it \textbf{Определение 4. Ковариация} двух случайных величин $X$ и $Y$ это
$$Cov(X, Y) = \E (X - \E X)(Y - \E Y).$$
}

{\it \textbf{Определение 5. Дисперсия} случайной величины $X$ это $\D (X) = Cov(X, X)$.}

{\it \textbf{Определение 6. Корреляция} двух случайных величин $X$ и $Y$ это
$$corr(X, Y) = \frac{Cov(X, Y)}{\sqrt{\D X \cdot \D Y}}.$$}

\z{} Докажите, что корреляция может принимать значения только в отрезке $[-1, 1]$. Т. е. для любых $X$ и $Y$ верно: $-1 \leqslant corr(X, Y) \leqslant 1$.


\end{document}
